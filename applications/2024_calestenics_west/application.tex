\documentclass{../../templates/amendment}
\usepackage{blindtext}
\usepackage[ngerman]{babel}
\usepackage{graphicx}
\usepackage{subcaption}
\usepackage[skip=2pt,font=small]{caption}

% \usepackage{showframe}
\occasion[Berzirksvertretung]{Bezirksvertretung Innenstadt-West}
\date{29. Dezember 2024}
\setcounter{nthantrag}{1}
\begin{document}

\begin{boxed}{Sportanlage Am Tremonipark / Haldenstr.}{Jan Lucas Brause, Maximilian Sackel}
    Die neue Calisthenicsanlage im Tremoniapark bildet in Kombination mit dem Basket- und Fußballplatz einen Treffpunkt für alle Generationen und den Querschnitt der Bevölkerung.
    Neben spielenden Kindern auf dem Bolzplatz und Jugendlichen auf dem Basketballplatz finden sich dort genauso Erwachsene, die ihre Zeit an der Calisthenicsanlage zum Trainieren nutzen.
    Ebenso wird die Sportanlage tagsüber von vielen Studierenden und jungen Berufstätigen als unverbindliche und kostenlose Trainingsmöglichkeit äußerst gut angenommen.
    Zudem haben sich bereits Sportvereine gebildet, welche diese Sportanlage für ihren Trainingsmittelpunkt nutzen und somit das angeleitete Sportangebot und die sozialen Strukturen vor Ort stärken.

    Die Nutzung der Sportanlagen ist in den Wintermonaten zu Nachmittags- und Abendstunden jedoch nicht möglich, da bei frühem Einbruch der Dunkelheit die Anlage aufgrund fehlender Beleuchtung komplett in der Finsternis liegt.
    Dies führt dazu, dass die Sportanlage weitestgehend von Sportlern und Spielenden ungenutzt bleibt und zu einem Angstraum wird.
    Geführte Schreckschusswaffen, Drogendelikte oder Wohnungslose sind im Schutz der Dunkelheit dort nicht selten anzutreffen.
    Auch Vandalismus an den Sportstätten kann durch die Installation von Beleuchtung und der damit einhergehenden längeren Nutzung vorgebeugt werden.
    Das umliegende dichte Unterholz wird regelmäßig von Abseits der Wege gehenden Personen aufgesucht.
    Tauchen diese auf einmal in der Dunkelheit aus den Gebüschen auf, sorgt dies regelmäßig für große Schockmomente.
    Der unbeleuchtete Standort ist somit besonders für Schutzbedürftige unattraktiv.

    Eine Beleuchtung im Zeitschaltbetrieb bis beispielsweise 21:00~Uhr verlängert die Nutzungsdauer und sorgt durch die Anwesenheit von
    Trainierenden und Sportlern dafür, dass die Fläche unattraktiver für benannte dubiose Tätigkeiten wird.
    Des Weiteren wird durch eine Beleuchtung in den Abendstunden ein ausgeweitetes Sportangebot für zeitlich und finanziell eingeschränkte Bürger und Bürgerinnen geschaffen.

    Die Lichtverschmutzung lässt sich durch eine Zeitschaltung auf die Nutzungsdauer minimieren.
    Eine weitere nennenswerte Beeinträchtigung der Fauna, durch das Licht, ist aufgrund der Menschen und Hunde welche bereits regelmäßig das Unterholz durchkämmen nicht zu erwarten.
    Eine Störung der Anwohner aufgrund der Beleuchtung kann aufgrund der Entfernung zu Wohngebäuden ausgeschlossen werden.
    Bei einer ähnlichen Spiel- und Sportanlage an der Düsterstraße in Körne wurde bereits eine Beleuchtung ohne das Verlegen von Stromleitungen im Erdreich mittels einer integrierten Solareinheit realisiert.
    Ähnliche Maßnahmen wären auch an beschriebener Stelle wünschenswert.

    Zudem möge die Bezirksvertretung beschließen im Sinne der beschlossenen Richtlinie (EU) 2020/2184 des Europäischen Parlaments und des Rates vom 16.~Dezember~2020 über die Qualität von Wasser für den menschlichen Gebrauch (ABl. L~435 vom 23.12.2020, S.~1) zur Bereitstellung von kostenlosem Trinkwasser im öffentlichen Raum einen Trinkwasserbrunnen in direkter Nähe zu den Sportstätten aufzustellen.
    Der Gesetzentwurf setzt die Regelung nach Artikel~16 Absatz~2 Satz~1 der Richtlinie um, wonach die Mitgliedstaaten sicherstellen, dass Leitungswasser zur Nutzung als Trinkwasser an öffentlichen Orten durch Innen- und Außenanlagen bereitgestellt wird, soweit dies technisch durchführbar und unter Berücksichtigung des Bedarfs und der örtlichen Gegebenheiten, wie Klima und Geografie, verhältnismäßig ist (Paragraf~50 Absatz~1 Satz~2 WHG~neu).

    \pagebreak
    Im Anhang sind Abbildungen für eine bessere Einsicht der Sachlage angefügt.
    Einen Ortstermin zu einer Machbarkeitsstudie, sowie die Bündelung weiterer möglicher Verbesserung und Sanierungsvorschläge, halten wir für sinnvoll.

    Über eine Bearbeitung und positive Rückmeldung würden wir uns sehr freuen.
    Selbstverständlich stehen wir gerne für Rückfragen und zur Terminvereinbarung zur Verfügung.

    \begin{figure}[htpb]
        \centering
        \begin{subfigure}[t]{0.3\textwidth}
            \begin{center}
                \includegraphics[width=\linewidth]{pictures/photo1.jpg}
                \caption{Dicht bewachsener Weg, der passiert werden muss um zu den Sportanlagen zu gelangen.}
            \end{center}
        \end{subfigure}
        \begin{subfigure}[t]{0.5\textwidth}
            \begin{center}
                \includegraphics[width=\linewidth]{pictures/photo5.jpg}
                \caption{Übersicht der räumlichen Gegebenheiten. Rechts der Fußballplatz, in der Mitte der Basketballplatz und hinten links die Calisthenicsanlage.}
            \end{center}
        \end{subfigure}
        \caption{Die gesamte Anlage ist von dichtem Unterholz umgeben. Der Charm des Trainings im Grünen in den Sommermonaten, schirmt im Winter jegliches Streulicht von benachbarten Gebäuden ab, sodass die gesamte Anlage finster in den Abendstunden wird.}
    \end{figure}

    \begin{figure}[htpb]
        \centering
        \begin{subfigure}[t]{0.32\textwidth}
            \begin{center}
                \includegraphics[width=\linewidth]{pictures/photo2.jpg}
                \caption{Basketballplatz}%
            \end{center}
        \end{subfigure}
        \begin{subfigure}[t]{0.32\textwidth}
            \begin{center}
                \includegraphics[width=\linewidth]{pictures/photo3.jpg}
                \caption{Calisthenicsanlage}%
            \end{center}
        \end{subfigure}
        \begin{subfigure}[t]{0.32\textwidth}
            \begin{center}
                \includegraphics[width=\linewidth]{pictures/photo4.jpg}
                \caption{Fußballplatz}%
            \end{center}
        \end{subfigure}
        \caption{Sportstätten, die in den Wintermonaten in den Abendstunden aufgrund der fehlenden Beleuchtung nicht mehr genutzt werden können und einen Angstraum bilden, sobald nicht von aktiven Sportlern besucht.}
    \end{figure}

    \begin{figure}[htpb]
        \centering
        \includegraphics[width=0.8\linewidth]{pictures/map3.pdf}
        \caption{Übersicht zur räumlichen Situation der unbeleuchteten Fläche mit Vorschlägen für geeignete Standorte von Lampen \textbf{L1} bis \textbf{L11} sowie eines Trinkwasserspenders \textbf{B1}.}%
        \label{fig:}
    \end{figure}
\end{boxed}

\end{document}
