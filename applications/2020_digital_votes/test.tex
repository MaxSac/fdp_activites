\documentclass{../../templates/amendment}
\usepackage{blindtext}
% \usepackage{showframe}
\occasion[Ortsparteitag]{Ortsparteitag der FDP Dortmund-Ost}
\date{18. August 2020}
\begin{document}

\begin{boxed}{Digitale Abstimmungen}{Samir El Hammouti, Maximilian Sackel}
Wir Freie Demokraten fordern einen Digitaliserungsparagraph in die Satzung des
Kreisverbandes Dortmund aufzunehmen.
Dieser soll Mitgliedern die Möglichkeit einer digitalen Stimmabgabe ermöglichen.
Dabei soll das Stimmrecht sowohl bei physischer Anwesenheit, als auch durch
physische Abwesenheit möglich seien (siehe NRW-Landesssatzung \S 15.7).
Personen, welche ihr Stimmrecht aufgrund der digitalen Form nicht wahrnehmen können,
sollen durch Angabe einer kurzen Begründung eine analoge ebenbürtige Alternative
zur Verfügung gestellt bekommen.

Die Abstimmungsergebnisse sind in einer geeigneten Form im Anschluss der Abstimmung
für einen längeren Zeitraum für alle Mitglieder sichtbar zur Verfügung zu stellen.

Des Weiteren soll eine bindende Abstimmung auch über einer längere Periode möglich
sein, damit mehr Mitglieder von ihrem Stimmrecht Gebrauch machen können.

Die Prüfung durch den Kreishauptausschuss (\S~10 Zeile 2) soll bei Ortsverbänden
auch im Anschluss der Abstimmung ermöglicht werden.

Des Weiteren ist bei Anträgen an übergeordneten Institutionen ein nicht bindendes
Stimmungsbild der Mitglieder des Kreisverbandes über ein begrenzten Zeitraum
einzuholen, um gegebenenfalls diese im Sinne der Mitglieder auszubessern.


Stichpunkte Kreissatzung:
\begin{itemize}
    \item[\S 12.3] Kreisparteitag organ Antraege zu verabschieden -> ordentliche
        Kreisparteitag findet alljaehrlich im ersten Kalendervierteljahr statt
        -> worst case nur jedes jahr neue antraege
\end{itemize}

Stichpunkte Landessatzung:
\begin{itemize}
    \item[\S 15.4] schrichtlich und geheim
    \item[\S 16.2] same as every
\end{itemize}
\end{boxed}

\end{document}
